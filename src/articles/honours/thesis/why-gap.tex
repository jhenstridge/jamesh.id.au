% -*- mode: latex; mode: auto-fill -*-
\chapter{Features of \GAP}

\section{Requirements}

When deciding on what language to use for the simulation system, there
were a number of features that we were looking for.  

\subsection{Features}

When choosing the language to implement GASP in, there were a number
of required and desired features we looked for:

\begin{itemize}
\item \textbf{The language must allow rapid development}.  It is
important that the language be quick to use, so that end users of GASP
can easily experiment with the algorithms they are working on.  This
requirement is most easily met by using an interpreted language.
\item \textbf{must be easy to learn and have a clear syntax}.
Ideally, it should be very easy for a new user to start extending GASP
to handle new algorithms.  Also, programs written in the language
should be easy for others to be able to read without a full knowledge
of the language.
\item \textbf{must be fast and stable}.  Programs written in the language must
run fast.  Stable means that programs written in the language
should not crash half way through a simulation, causing all
information up to that point to be lost.
\item \textbf{should have some graphics capabilities}.  It
would be nice to be able to design a graphical user interface to run a
simulation in.
\item \textbf{should make it easy to debug programs}.  The language
should make it easy to debug errors.  That is, when an error occurs,
rather than just exiting, it would be nice if it were possible to find
out what went wrong.  This feature is more important to people trying
to extend the GASP package, rather than those using existing
algorithms.
\end{itemize}


\subsection{The Choice of Language}

The \GAP{} language was chosen because it met these criteria.  S-Plus
was not chosen because it has had problems with long running
programs in the past, where its memory consumption would increase with
time, and S-Plus occasionally crashes under some conditions.

Another language that was considered was Python \cite{python}.  Python
shares a number of these desirable features with \GAP, but during
tests it was found to be slower than \GAP.  This is most likely caused
by the different types of memory management performed by the two
languages.  Also, the graphical capabilities of \GAP{} are more
convenient for use in GASP than those available for Python.

A more detailed summary of the relevant features of \GAP{} follows in
the next section.


\section{Relevant Features of \GAP}

The aim of this section is to highlight some of the features of the
system \GAP{}, which we used to develop GASP.  This section is not
intended to be a substitute for the \GAP{} documentation or as a
tutorial for the language.  Rather, it should familiarise the reader
with some of the basic concepts that we use in the remaining chapters.

\GAP{} stands for \emph{Groups}, \emph{Algorithms} and
\emph{Programming}.  Development started in 1985 at Lehrstuhl D f\"ur
Mathematik, RWTH-Aachen by a group led by Professor Joachim
Neub\"user.  Since then it has gone through a number of rewrites, and
is now maintained by an international group of developers coordinated
from Saint Andrews.  Although it was not specifically designed for the
types of statistical problems we considered, a number of its features
made it a very useful tool for our project.  \GAP{} is freely
available from the program's web site \cite{gap-www}.

Once installed, \GAP{} can be started from the shell prompt with the
following command:

\begin{verbatim}
$ gap
\end{verbatim}
%$ this dollar sign is to not confuse the syntax highlighting

This will start up \GAP{}, and print a \texttt{gap>} prompt where you can
enter \GAP{} commands to execute.

\subsection{General Description}

\GAP{} was designed as a language for use by mathematicians, so has a
syntax that corresponds fairly simply to written mathematics.  This reduces
the perceived learning curve required to use the system, which could
be a deciding factor for a new user when considering whether or not to
use GASP.

\GAP{} consists of a number of components centered around a small
kernel which is written in C.  The kernel is implemented in the main
\GAP{} executable, and is comprises:
\begin{description}
\item[Language] - the parser for the \GAP{} language combined with the
interpreter which is responsible for executing \GAP{} programs.  This
also includes the base of the \GAP{} object model, as described in
Section \ref{sect:gap-object-model}.
\item[Memory management] - the code that handles all memory allocation
and deallocation in \GAP.  \GAP{} uses a form of garbage collection,
so users do not have to concern themselves with explicitely allocating
or freeing memory in their program.
\item[Selected objects and operations] - a number of functions and
data types which are impossible or very inefficient to implement in
the \GAP{} language.  These include implementations of the base data
types such as integers, lists and some other data types, along with
some of the basic operations for those types.
\end{description}

On top of this kernel are the standard library, some data sets and the
documentation.  The standard library takes the primitives defined
inside the kernel and builds them up to create data types that can be
used to work on various mathematical problems.

There are a number of data sets (the small groups library, etc)
distributed with \GAP{}, which can be accessed from the language.  These
are not relevant to the problems we are interested in here.

\GAP{} contains a builtin documentation system, so the interpreter can
extract a section from a specially formated \TeX{} documentation file
for a particular function.  This is then displayed to the user, the
format depending on the way \GAP{} was invoked.  This yields a very
simple way to provide both online help for the functions in \GAP{},
and produce a high quality typeset version of the library reference.

In addition to the standard library, \GAP{} provides a way for any
user to contribute packages known as \emph{share packages} to be used.
The use of share packages makes it very easy to package together a
number of functions and data structures together for distribution.
Official share packages are refereed.

For GASP, most of the standard library and all the data sets were not
used.  However, a lot of the infrastructure these components were
built on was used.


\subsection{Interpreted Nature}

\GAP{} is an interpreted language.  If a language is not interpreted,
a user writes a program, then compiles the program to machine code,
executes the program to test it, and debugs it if any errors occur.
If errors are found, the user must recompile and test the application
again.  With an interpreted language however, the compile stage of the
normal \texttt{write $\to$ compile $\to$ test $\to$ debug} cycle gets
removed, which speeds up development.

\GAP{} is also \emph{interactive}. This means that a user can enter
a command at the \texttt{gap>} prompt, \GAP{} evaluates the command
and prints the result.  This is the usual \texttt{read $\to$ eval
$\to$ print} loop found in many interpreted languages such as python
or the UNIX shell.

Often, interactive languages reduce the experimentation part of the
development process down to just \texttt{test $\to$ debug}.  The above
simplification of the development cycle leads to fast experimentation
and fast development when combined with the fact that memory
management can be ignored.

Of course, the interpreted nature of \GAP{} can lead to less than
optimal speed.  In most cases, the tradeoff against shorter
development time is sensible.  In the cases it is not sensible, a
\GAP{} to C translator is available that allows compilation of \GAP{}
code for extra speed.


\subsection{\GAP{} Features}

As well as the many features related to group theory, \GAP{} contains
many features found in most general purpose languages.  Many of these
were necessary to build up the mathematical features of the language.
These include a number of data types including some aggregate data
types, and many of the normal control structures found in most
programming languages.


\subsubsection{Basic Data types}

As well as the more specialised group theory types found in the
standard library, there are a number of general purpose types.  These
include integer, rational and floating point numbers.  Integers and
rational numbers convert to an arbitrary precision representation
if they grow too large.

At the time of writing, the floating point support is not well
integrated into the \GAP{} system.  While it is possible to perform
binary operations on two floats, sometimes an error occurs when the
same operation is applied to a float and a non-float.  Also, not all
functions accept a float where some other type of number is
expected due to these problems.  It is expected that these problems
will be resolved in the future.

In addition to these basic types, \GAP{} provides a number of aggregate
data types.  The main ones are lists and records.

Lists in \GAP{} can be \emph{heterogeneous}.  That is, a list can contain
elements of more than one type.  Lists can be expanded and contracted
as needed at runtime using the \texttt{Append} and \texttt{RemoveSet}
functions.  You can perform complex indexing operations on a list.
For example, if we have a list initialised as
\verb|list := [7,8,9,10,42,11]|, then the expression
\verb|list{[1,3,5]}| returns a list containing the first,
third and fifth elements of the initial list, which is \verb|[7,9,42]|.

Records in \GAP{} are similar to \texttt{struct}s in C or
\texttt{record}s in Pascal.  They provide a way to group a number of named
variables together.

These aggregate data types can be nested to produce more complex types
as required.  \GAP{} also provides a way to \emph{objectify} lists and
records, which gives you even more control over how they are handled.
This is covered in the next section.


\subsubsection{Control Structures}

\GAP{} provides many of the standard control structures found in most
languages.  These include the standard \texttt{if .. then .. else
.. end if} conditional statements, as well as statements to iterate
over the members of a list.  There are also the more general
\texttt{while} and \texttt{repeat .. until} loops.

For more information, see the \GAP{} reference manual \cite{gap-ref}.

\subsection{The \GAP{} Object Model}\label{sect:gap-object-model}

\GAP{} provides the ability to create hierarchies of types similar to what
is found in many object oriented languages.  We can create instances
of these types by \emph{objectifying} records or lists.

The main reason for creating these objectified types is \GAP{}'s
operation system.  Operations are analogous to methods in most object
oriented languages and provide a way of creating \emph{polymorphic}
functions.  That is, functions that perform different actions
depending on the types of the arguments.

\GAP{} also provides the ability to overload the standard operators.
This means that you can implement the standard operators such as $+$,
$-$, $*$, $=$ and $\le$ for any new object types created in \GAP{}.
Operator overloading in \GAP{} uses the operations system described
above.  This means that we can provide different implementations of
the operations depending on the types of the operands.

Although \GAP{} appears to be purely procedural, the combination of
objectified types and operations gives as much power as found in more
traditional object oriented languages.  The difference in syntax is
not expected to be a problem, as most users of GASP will probably not
have much experience in object oriented languages to start with.


\subsection{Functions}

\GAP{} treats functions like any other variables.  They can be passed as
an argument to another function, or returned from a function.  In
fact, the usual way of defining a new function is through an
assignment to a variable.

Similar to functions in languages like Pascal, a \GAP{} function has
access to the variables in the environment where it was defined.  If
we define one function within the body of another function, it will
have access to the local variables of the enclosing function as well
as those in the \emph{global name space}.  A \emph{name space} refers
to a mapping between variable names and their corresponding values.
Usually a particular name space is only used in certain contexts, but
variables in the global name space are available from all parts of the
program.

When combining access to the defining environment of a function with
passing function variables as arguments to other functions, it is not
quite clear what a language should do.  Consider the following
fragment of \GAP{} code:

\begin{lstlisting}{}
g := function()
    local i, h;
    i := 0;
    h := function();
        i := i + 1;
        return i;
    end;
    return h;
end;
f := g();
\end{lstlisting}

This creates the function \texttt{f}.  When we exit from the call of
\texttt{g}, the local variables that make up the environment in which
the function \texttt{h} was defined are preserved.  The other common
behaviour in this situation is to not give the function \texttt{f}
access to \texttt{g}'s local variables, or not preserve their values
across calls to \texttt{g}.  \GAP's behaviour was very convenient when
writing parts of GASP.

Successive calls to \texttt{f} will result in a series of positive
integers being returned.  If we create another copy of \texttt{f} by
calling \texttt{g} again, the series returned by the copy will be
independent of the first \texttt{f}.  In effect, each \texttt{f} has a
copy of the variable \texttt{i}.

Such an \texttt{f} is known as a \emph{closure}.  That is, it is a
combination of the function and the environment it was defined in.

We used this functionality fairly extensively inside GASP.


\subsection{\XGAP}

\XGAP{} is a combination of a share package for \GAP{} along with a
wrapper front end that provides graphical capabilities using the X
Window System.  \XGAP{} can be started by running the \texttt{xgap}
command.  This will open an X window for the console and load up the
share package part of \XGAP{}.

The \texttt{xgap} share package provides the ability to create
\emph{graphic sheets} which act like a canvas on which various
geometric figures such as rectangles, lines and circles can be drawn.
These graphic sheets handle all redraws for the canvas, so a program
does not need to worry about redrawing the shapes after the canvas
gets obscured.  \XGAP{} also provides the capability to save the
contents of a graphic sheet to an encapsulated PostScript (EPS) file
which can then be printed or included in a document, as I have done so
in Figure \ref{fig:red-black-output}.

Graphic sheets can also take mouse input from the user, which is
passed back to the program, and add extra menu items to the menu bar at
the top of the graphic sheet.

\XGAP{} opens a new window when you ask for help on a particular
topic, and provides some cross referencing support so you can click on
other topic names in the help window.

We use \XGAP{}'s graphics abilities for visualisation in GASP.


\subsection{More Information}

This section is not meant to be a complete description of the \GAP{}
system.  Comprehensive documentation is distributed with the \GAP{}
system, consisting of tutorial, programming, extension and reference
manuals and online help.  The tutorial \cite{gap-tut} and reference
\cite{gap-ref} manuals are good places to start learning about \GAP{}.

Online help is available for almost all \GAP{} functions by typing
\texttt{?}\textsl{function\-name} at the \texttt{gap>} prompt.  \GAP{} also
supports tab completion, which can be used to find appropriate
functions.  By typing in part of a function name, and pressing the
\texttt{tab} key, either the name is completed or a list of
names matching what was typed is listed.

\subsection{Problems With \GAP}

When working with \GAP{}, there were a few problems that surfaced.
The first was the lack of floating point support in the released
version.  For work on GASP, I was able to obtain the unreleased
development version of \GAP{}, which has some floating point support
built in.  As floating point support is not very well integrated, into
\GAP{} at present, its use inside GASP has been quite limited.

The other problem with \GAP{} is the lack of multiple global name
spaces.  This prevents the programmer from creating functions that are
private to a particular share package.  Although this did not cause
any big problems when developing GASP, it means that people extending
GASP have to be careful not to use function names that may be used in
other \GAP{} packages, as it may produce unpredictable results.
Multiple global name space support is another feature that is planned
for \GAP{} in the future.

Although \GAP{} has a number of problems, overall it proved to be a
useful tool to use as a base for GASP.