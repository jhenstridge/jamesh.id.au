% -*- mode: latex; mode: auto-fill -*-

\chapter{Introduction}

\section{Aims}

In this project, I looked at the problem of implementing a framework
for simulating processes related to spatial statistics using
algorithms such as the Metropolis-Hastings algorithm.

The aim of the project was to produce a framework that could handle as
many different algorithms in the class as possible, while remaining
simple enough to be understood and easy to use.  The result is the
GASP package \cite{gasp-source}, which stands for
\emph{\textbf{G}enerating \textbf{A}lgorithms for \textbf{S}patial
\textbf{P}atterns}.

\section{The Choice of Language}

Simulations of this type often require a lot of computation, so have
traditionally been implemented in languages such as C or C++, due to
the speed and flexibility they give.  Using such a language has
drawbacks:

\begin{itemize}
\item a program would be written specifically for a particular
problem, so may require substantial modification to cover a different
problem.
\item knowledge of the C language, which includes many details, such
as memory management, that are not directly related to the problem.
This significantly raises the barrier for someone wanting to use such
a program.  Also, C is perceived as being difficult to learn, which
reduces the value of a system which is aimed at mathematicians.
\item as C is a compiled language, it takes more effort to make a
change to the algorithm, which can slow down development.
\end{itemize}

Rather than having many programs, each implementing their own
algorithm, it would be better to have a framework from which many
different algorithms could be explored with.  That was the aim of
my project.

An interpreted language was chosen because of reduced development
overhead, allowing more rapid experimentation and implementation of
new algorithms.

\section{\GAP}

The language chosen for the implementation was \GAP{} \cite{gap-www}.
While \GAP{} was originally designed for work with group theory, a
number of its features made it an interesting choice for this class of
problem.

While S-Plus \cite{s-plus} may be considered the obvious choice of
language, it suffers from some problems when used to run simulations
that take a long time.  Problems related to instability and memory
leaks are magnified during long runs, which may lead to loss of data
if the interpreter crashes.  Also, to someone who does not already
know S-Plus, the syntax can be difficult to learn.

Like S-Plus, \GAP{} is an interpreted language.  Its syntax was
designed with mathematicians in mind, so is quite easy for a new user
to pick up.  \GAP{} has some object oriented features, which were used
when implementing the GASP framework.

As well as having the rapid development advantages of interpreted
languages, \GAP{} also has interactive debugging features, and support
for designing graphical user interfaces that consist of menus and
geometric shapes drawn on a canvas.

Using \GAP{}'s ability to pass functions as arguments to other
functions, GASP allows the user to hook into many different parts of
the simulation framework.

\section{The Change Log}

All the algorithms that I considered involved transitions between
different states of a configuration of some type of object such as
points in the plane.  Quite often, it is these transitions, or changes
in the configuration, that are interesting.

In the GASP framework, these changes form an important part of the
simulation.  Borrowing some ideas from the undo capabilities found in
some software, all changes made to the configuration are encapsulated
inside objects that hold all information to apply and revert the
change in state.

As the amount of memory required to encode these these changes is
generally smaller than the amount required for the configuration
itself, it was feasible to log a lot more information about what
happened during the simulation than with systems where the complete
state of the configuration was logged.  This unique feature of GASP
gives users the ability to analyse a simulation much more effectively,
even after the simulation has completed.

\section{Chapter Guide}

The second chapter gives a brief overview of the mathematics behind
some of the problems that the GASP framework covers.

The third chapter covers the features of \GAP{} that were useful when
designing GASP, and more detailed reasons why it was chosen as the
language to implement GASP in.

The fourth chapter is an introduction to the functionality provided by
the GASP framework.  This chapter is aimed at users who are interested
in using the existing functionality in GASP, rather than extending it
to handle new algorithms.

The fifth chapter looks at the implementation of the framework, and
how to extend its functionality -- both the algorithms that are
supported and the types of configurations that can be used in
simulations.

