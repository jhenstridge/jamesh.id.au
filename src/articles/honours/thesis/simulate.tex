\documentclass[12pt,a4paper]{book}  % -*- mode: latex; mode: auto-fill -*-
\usepackage{epsfig}
\usepackage{listings}

\begin{document}

\lstdefinelanguage{Gap}%
  {keywords={while,for,in,do,od,repeat,until,if,then,else,elif,fi,
      function,end,local,return,true,false,rec,quit,break},%
  sensitive=true,%
  commentline={\#},%
  blankstring=true,%
  indent=2em,%
%  frame=tlbr,%
  stringizer=[b]{"'}%
}%
\lstset{language=Gap}

\title{Simulation system chapter}
\author{James Henstridge}

\chapter{Simulation System}


\section{Background}

When designing the simulation system for gasp, the main goal was to
have a framework that would cover most simulations that people were
likely to try while being easy to use, and minimize the work the
user needed to do.

Most of the simulations we looked at followed the following general
loop for its iterations shown in Figure \ref{fig:flowchart}

\begin{figure}
\label{fig:flowchart}
\centering
\epsfig{file=simulate-flowchart.eps,width=4in}
\caption{Simulation flow of control}
\end{figure}

The main simulation specific to note in the general algorithm are:

\begin{itemize}
\item The representation for the state
\item The rules used to propose a new state
\item The rule used to calculate the acceptance probability for the
new state.
\end{itemize}

For example, in the Strauss process, the state is the collection of
points in the configuration.  The proposal rules are to choose at
random between creating a new point and deleting an existing point.
The rule for calculating acceptance probability is as normal.

The design of the simulation system followed this model fairly
closely.

\section{The Configuration}

The configuration object represents the current state of the system.
It is mainly a collection of the objects that make up the state.  In
order to give maximum flexibility to the framework, I made the
Configuration object an abstract base type -- you need to create a
subtype in order to actually use the simulator.  This allows changes
to the way a configuration is implemented without affecting
surrounding code.

\begin{lstlisting}{}
# the picture object configuration ...
DeclareCategory("IsConfiguration", IsObject);
BindGlobal("ConfigurationFamily",
           NewFamily("ConfigurationFamily",
           IsConfiguration));
\end{lstlisting}

The configuration is composed of objects.  In the same way, we create
an abstract type for these objects.  The name PictureObject was chosen
because Object is already taken as the base category in GAP.  The
objects in the configuration may be points, lines, discs or something
else.

\begin{lstlisting}{}
# declarations of the picture object type ...
DeclareCategory("IsPictureObject", IsObject);
BindGlobal("PictureObjectFamily",
           NewFamily("PictureObjectFamily",
           IsPictureObject));
\end{lstlisting}

As these types are abstract, we need some standard ways to manipulate
the configuration and perform some simple calculations on the objects
or the whole configuration.  These are implemented through GAP
operations.

\subsection{PictureObject Operations}

One of the main calculations we want to perform on the objects in the
configuration is calculation of the distance between two objects.
As the released versions of GAP at the time of writing had no square
root support for floating point numbers, the choice was made that the
operation should return the squared distance between the two:

\begin{lstlisting}{}
DeclareOperation("SquareDistance",
                 [ IsPictureObject,
                   IsPictureObject ]);
\end{lstlisting}

As a number of the algorithms use the idea of marking objects in the
configuration as part of the simulation, a standard API is provided
for this:

\begin{lstlisting}{}
DeclareOperation("AddMark", [ IsPictureObject,
                              IsString ]);
DeclareOperation("DelMark", [ IsPictureObject,
                              IsString ]);
DeclareOperation("HasMark", [ IsPictureObject,
                              IsString ]);
\end{lstlisting}

As most PictureObject types will be implemented as objectified
records, standard implementations of these operations are provided.
All that is required is that the constructor for the object sets the
attribute marks to an empty list.  If the internal structure of the
object does not match these assumptions, alternative implementations
of these operations can be provided with the PictureObject type.

The marks API provides a way to set boolean flags on an object.  Some
mark names may have special meanings.  For example, using colour names
as the mark names may be used to colour the object when visualising
the simulation.

Also, in order to visualise the configuration, we need some way to
draw the PictureObject.  This is handled by the DrawObject operation:

\begin{lstlisting}{}
DeclareOperation("DrawObject", [ IsPictureObject,
                                 IsGraphicSheet ]);
\end{lstlisting}

The second argument is a standard XGAP graphic sheet object.  The
DrawObject operation should return the XGAP object that was placed on
the GraphicSheet, so that it can be further manipulated
(eg. recolouring the object, deleting it, etc).

Additional operations may be added for specific object types to help
in particular simulations without much trouble.  It is also fairly
easy to add new standard operations, although the amount of work
required to add new operations increases with the number of object
types.

\subsection{Configuration Operations}

We would also like to make the implementation of the algorithms in a
simulation independent of the actual implementation of the
configuration as well as the objects.

The configuration is basically a set of objects, holds some
information about the bounds for displaying the configuration, and may
be some extra state information for the simulation.

The first couple of operations on the Configuration are for adding and
removing objects in the configuration:

\begin{lstlisting}{}
DeclareOperation("AddObject", [ IsConfiguration,
                                IsPictureObject ]);
DeclareOperation("DelObject", [ IsConfiguration,
                                IsPictureObject ]);
\end{lstlisting}

Most calculations can easily be performed if we have a list of the
objects in the configuration.  But for some representations creating a
list of objects may be time fairly time consuming.  Also, some of the
calculations we want to do on may be more efficient with the
configuration's native representation.

For instance, the configuration may `bin' the objects into different
lists which can make for faster distance calculations.  Or it might
sort the objects based on some mark, to speed up calculations that
look for objects with a particular mark.

For this reason, we need some standard operations to perform these
calculations so that the most efficient way of completing them for a
particular Configuration representation.  Currently, the following
operations are defined:

\begin{lstlisting}{}
DeclareOperation("ConfigWindowArea",
                 [ IsConfiguration ]);
DeclareOperation("CountObjects",
                 [ IsConfiguration ]);
DeclareOperation("CountObjectsWithMark",
                 [ IsConfiguration, IsString ]);
DeclareOperation("GetObjectsWithMark",
                 [ IsConfiguration, IsString ]);
DeclareOperation("CloseObjects",
                 [ IsConfiguration, IsPictureObject,
                   IsScalar ]);
DeclareOperation("CloseObjectsWithMark",
                 [ IsConfiguration, IsPictureObject,
                   IsScalar, IsString ]);
\end{lstlisting}

These operations perform the following actions:

\begin{description}
\item[ConfigWindowArea] returns the size of the window in which the
objects are placed.
\item[CountObjects] counts the objects in the configuration.
\item[CountObjectsWithMark] counts the objects in the configuration
for which a particular mark has been set.
\item[GetObjectsWithMark] returns a list of the objects with a
particular mark set.
\item[CloseObjects] counts the objects in the configuration that are
within a certain distance from the given object.
\item[CloseObjectsWithMark] similar to \emph{CloseObjects}, except it only
examines objects with a particular mark set.
\end{description}

We also want some way to pick a random object in the configuration.
This is handled by the \emph{ChooseRandomObject} operation:

\begin{lstlisting}{}
DeclareOperation("RandomNewObject",
                 [ IsConfiguration, IsFunction ]);
\end{lstlisting}

Next, to aid in the visualisation, we have a \emph{DrawConfiguration}
operation:

\begin{lstlisting}{}
DeclareOperation("DrawConfiguration",
               [ IsConfiguration, IsGraphicSheet ]);
\end{lstlisting}

Lastly, the algorithms need some way to create new objects to add to
the configuration.  So we have another operation to create a new
object:

\begin{lstlisting}{}
DeclareOperation("RandomNewObject",
                 [ IsConfiguration, IsFunction ]);
\end{lstlisting}

The second argument is a function that should return a random number
between zero and one.  This allows the person implementing the
algorithm to experiment with different random number distributions.
If you just want a uniform distribution, you can pass in the following
function:

\begin{lstlisting}{}
RandRat := function()
    return Random([0 .. 10000])/10000;
end;
\end{lstlisting}

\section{Basic Simulation Algorithm}

Now the flow of the simulation can be represented as pseudo code like
so:

\begin{lstlisting}{}
local cnf, prob, iter;

cnf := CreateConfiguration();
for iter in [ 1 .. num_iterations ] do
  ProposeChangeToConfiguration(cnf);
  prob:= ProbabilityOfAcceptingChange(cnf);
  if RandomNumBetween0and1() <= prob then
    AcceptChange(cnf);
  else
    UndoChange(cnf);
  fi;
od;
\end{lstlisting}

So we need some way of handling changes to the configuration in such a
way that they can be backed out if the proposed state is not accepted.
This leads on to the development of the idea of Change objects.

\section{Change Objects}

The two methods of handling the management of changes to the
configuration that I looked at were:

\begin{itemize}
\item not making changes to the configuration until (and only if) the
change is accepted.  This is fairly simple to implement, but means
that the user supplied code that handles the probability calculation
must take both the old state and the proposed change when performing
the calculation.
\item encapsulate the changes in some form of object which knows how
to apply and revert itself to the configuration.  This means that the
probability calculation function can be passed the new state, which
may make calculations easier.
\end{itemize}

I decided to use the second option, as it gives more flexibility, and
makes implementing some other features a lot easier.  For instance, by
keeping track of the changes made to the configuration, we can travel
back in time if we wish.  It is also very similar to one of the
patterns that is comonly used to implement undo in many computer
applications.

These change objects only require a very simple API.  The only actions
they need to perform is to apply or revert themselves on a
Configuration object.

\begin{lstlisting}{}
DeclareCategory("IsChange", IsObject);
BindGlobal("ChangeFamily",
           NewFamily("ChangeFamily", IsChange));

DeclareOperation("ApplyChange", [ IsChange,
                           IsConfiguration ]);
DeclareOperation("RevertChange", [ IsChange,
                           IsConfiguration ]);
\end{lstlisting}

It is also very useful if these Change objects implement the
\emph{PrintObj} operation so that we can simply \emph{Print()} them to
a file.  As with most GAP objects, the output of the \emph{PrintObj}
operation should be the function call used to construct the object.

As most simulations will want to perform fairly simple changes to the
Configuration.  For this reason, a number of standard Change objects
are made available.  This does not inhibit the creation of new types
of Changes, but does make things easier for the user if these changes
are appropriate.

\begin{lstlisting}{}
AddObjectChange := function(picobj);
DelObjectChange := function(picobj);
ReplaceObjectChange := function(addobj, delobj);
ReplaceMultipleObjectsChange :=
     function(addobjlist, delobjlist);
\end{lstlisting}

These standard change objects are implemented in terms of the
\emph{AddObject} and \emph{DelObject} methods, so can be used with any
configuration and object types.

With the definition of the change objects, we can now more clearly
define the \emph{ProposeChangeToConfiguration}
\emph{ProbabilityOfAcceptingChange} functions.

\begin{description}
\item[ProposeChangeToConfiguration] is a function that takes the
configuration as its argument and returns a Change object representing
the proposed change to the configuration.
\item[ProbabilityOfAcceptingChange] is a function that takes the
configuration and proposed change as arguments and returns the
probability of accepting the change.
\end{description}

So the user need only provide an initial configuration plus a proposal
and acceptance probability calculating function and the rest should be
the same for all simulations.

Using the fact that GAP allows you to pass functions as arguments to a
function, the main loop of the simulator can be wrapped up in a
function that takes this information as arguments.

%% XXXX change this for the new finish checking code
\begin{lstlisting}{}
Simulate := function(cnf, iter, propose, check, fd);
\end{lstlisting}

The first argument is the initial configuration for the system.  The
second is the number of iterations to run through.  The third and
forth arguments are the proposal and acceptance checking functions
mentioned above.  The last argument is a file name or output stream
object that a log of the simulation will be written to.

Posible options for \emph{fd} are \emph{OutputTextNone()} which
discards all logging information, \emph{OutputTextUser()} to write the
log to the user's terminal and \emph{OutputTextFile(``file name'',
true)} to write the log to a file.  Note that using
\emph{OutputTextFile} will result in better performance than just
passing in the string name, as it will mean GAP only has to open the
log file once.

Note that as the simulation doesn't save any state between iterations
except in the Configuration object, calling \emph{Simulate} once with
a count of 20 iterations is equivalent to two calls with 10
iterations.

This means that it is possible to build a function that updates an
XGAP \emph{GraphicSheet} every $n$ iterations by repeatedly calling
\emph{Simulate} with an iteration count of $n$ and then drawing the
configuration.

\section{Standard Algorithms}

Quite often the user will want to perform a number of simulations with
slightly different parameters to the algorithm.  It is a pain to have
to type out the function body for the algorithm, changing just a few
numbers in the process.  Also, it could become a source of error, if
the user mistypes the algorithm.

The closure support in GAP can help a lot with this.  Closures are
basically the combination of a function and its environment.  When a
function returns another function, it is creating a closure.  When the
returned function is executed, the environment it sees will be the one
where it was defined.

This means that it can access local variables of the function in which
it was defined, even though that function had returned.

Using closures, we can write functions that return functions
implementing a particular algorithm using the user's desired
parameters.

As an example, take the case of a proposal function that chooses to
add an object with probability $p$ and remove a random object with
otherwise.  Rather than making the user type out this common function
again each time a change to $p$ is made, we provide the following
`factory' function:

\begin{lstlisting}{}
CreateSimpleFlipPropose := function(prob)
    return function(cnf)
        local rn, point;

        rn := RandRat();
        if rn >= prob and CountObjects(cnf) > 0 then
            point := ChooseRandomObject(cnf);
            return DelObjectChange(point);
        else
            point := RandomNewObject(cnf, RandRat);
            return AddObjectChange(point);
        fi;
    end;
end;
\end{lstlisting}

Similar functions can be provided for other common cases.  Even in the
case of a particular algorithm, using `factory' functions like
this is beneficial, as it reduces code duplication.

A number of factories for common algorithms are provided in the
algorithms.gi file.  It is expected that this list will grow in the
future.

\section{Visualisation}

\emph{XXXX - write about the visualisation stuff here}

\section{Logging}

As a simulation is run, a log of what happens is created (provided you
don't use \emph{OutputTextNone()}).  The \emph{Simulate} function
writes out information about the proposed changes and whether they
were accepted or not.

After the simulation has run through (or if there is a crash while the
simulation is running), this log can be used to reconstruct the state
at any point throughout the simulation.  Thus, it can be used to
collect useful statistics, such as the number of points in the
configuration with respect to time, accept/reject ratios.  The
proportions of particular changes being picked and others.

As the string handling functions in GAP are fairly minimal, some of
these functions are probably best implemented in some other language,
such as perl or python.

I plan to write some simple scripts to generate graphs of some for
some of these statistics.

\end{document}
